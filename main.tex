\documentclass[11pt]{article}
\usepackage{amsmath}
\usepackage{amsthm}
\usepackage{amssymb}
%\usepackage{algorithmic}
\usepackage{algorithm}
\usepackage{subfig}
\usepackage{color}
\usepackage[english]{babel}
\usepackage{graphicx}
\usepackage{grffile}
%\usepackage{natbib}
%\usepackage[pdf]{pstricks}
\usepackage{wrapfig,epsfig}
%\usepackage{psfrag}
\usepackage{epstopdf}
\usepackage{bm}
\usepackage{url}
\usepackage{color}
\usepackage{epstopdf}
\usepackage{algpseudocode}
%\usepackage{scrextend}
%\usepackage[hidelinks,pdfencoding=auto,psdextra]{hyperref} %%% commet for arxiv
\usepackage[T1]{fontenc}
\usepackage{bbm}
\usepackage{comment}
\usepackage{dsfont}
%\usepackage{bbold}
%\usepackage{amsfonts}
%\usepackage[nottoc,numbib]{tocbibind}
%\settocbibname{References}
%\usepackage{smile}
\usepackage{IEEEtrantools}


 %%% print refs in table of contents
\let\C\relax
\usepackage{tikz}
\usepackage{hyperref}  %%% arxiv don't allow this.
\hypersetup{colorlinks=true,citecolor=red,linkcolor=blue} %%% Zhao : maybe we should comment this in submission.
\usetikzlibrary{arrows}
%\usepackage[lmargin=1in,rmargin=1in,tmargin=0.8in,bmargin=0.8in]{geometry}
\usepackage[margin=1in]{geometry}
%\linespread{1}
%\newcommand{\QED}{\hfill$\qed$}
\graphicspath{{./figs/}}

\newtheorem{theorem}{Theorem}[section]
\newtheorem{lemma}[theorem]{Lemma}
\newtheorem{definition}[theorem]{Definition}
\newtheorem{notation}[theorem]{Notation} 
%\newtheorem{proof}[theorem]{Proof}
\newtheorem{proposition}[theorem]{Proposition}
\newtheorem{corollary}[theorem]{Corollary}
\newtheorem{conjecture}[theorem]{Conjecture}
\newtheorem{assumption}[theorem]{Assumption}
\newtheorem{observation}[theorem]{Observation}
\newtheorem{fact}[theorem]{Fact}
\newtheorem{remark}[theorem]{Remark}
\newtheorem{claim}[theorem]{Claim}
\newtheorem{example}[theorem]{Example}
\newtheorem{problem}[theorem]{Problem}
\newtheorem{open}[theorem]{Open Problem}
\newtheorem{hypothesis}[theorem]{Hypothesis}
\newtheorem{question}[theorem]{Question}
%\fi

\newcommand{\wh}{\widehat}
\newcommand{\wt}{\widetilde}
\newcommand{\ov}{\overline}
\newcommand{\eps}{\epsilon}
\newcommand{\N}{\mathcal{N}}
\newcommand{\R}{\mathbb{R}}
\renewcommand{\P}{\mathbb{P}}
\newcommand{\RHS}{\mathrm{RHS}}
\newcommand{\LHS}{\mathrm{LHS}}
\renewcommand{\d}{\mathrm{d}}
\renewcommand{\i}{\mathbf{i}}
\newcommand{\norm}[1]{\left\lVert#1\right\rVert}
\renewcommand{\varepsilon}{\epsilon}
\renewcommand{\tilde}{\wt}
\renewcommand{\hat}{\wh}
\renewcommand{\eps}{\epsilon}
\newcommand{\diag}{\textrm{diag}}
\renewcommand{\d}{\mathrm{d}}
\newcommand{\sgn}{\mathrm{sgn}}



\DeclareMathOperator*{\E}{{\mathbb{E}}}
\DeclareMathOperator*{\Var}{{\bf {Var}}}
\DeclareMathOperator*{\Sup}{{\bf {Sup}}}
\DeclareMathOperator*{\Z}{\mathbb{Z}}
%\DeclareMathOperator*{\R}{\mathbb{R}}
\DeclareMathOperator*{\C}{\mathbb{C}}
\DeclareMathOperator{\supp}{supp}
\DeclareMathOperator{\poly}{poly}
\DeclareMathOperator{\nnz}{nnz}
\DeclareMathOperator{\rank}{rank}
\DeclareMathOperator{\new}{new}
\DeclareMathOperator{\pre}{pre}
\DeclareMathOperator{\old}{old}
\DeclareMathOperator{\tr}{tr}
\DeclareMathOperator{\T}{\top}
%\DeclareMathOperator{\mp}{mp}

\title{System Theory}
\author{TBD}
\date{August 2019}

\begin{document}

\maketitle
\begin{abstract}
    TODO
\end{abstract}

\section{Introduction}

%%% Zhao : The following thing is very irrevalent, I can't remember the purpose of the following thing.

%Given a directed graph $G(V,E)$, where $V$ is the set of vertices and $E$ is the set of edges. For every pair $u,v$, there is at most one directed between them.  Let $S$ denote a subset of $V$. We say vertex $u$ is the father of vertex $v$, if there is a directed path from $u$ to $v$. The goal is to output mapping $f : V \rightarrow \{0,1,2\}$ such that\\
%1. For each $v \in S$, $f(v) = 1$.\\
%2. For each $u,v$ if $u$ is the father of $v$, then $f(u) \leq f(v)$.


\section{Paging}

Competitive Paging Algorithms.

Amos Fiat, Richard M. Karp, Michael Luby, Lyle A. McGeoch, Daniel D. Sleator, and Neal E. Young

\url{https://www.cs.cmu.edu/~sleator/papers/competitive-paging.pdf}

\begin{algorithm}
\begin{algorithmic}[1]
\Procedure{\textsc{Paging}}{$k,n$}
    \State See the new variable $x( p_t, r(p_t,t) ) \leftarrow 0$ (It can only be increased at times $t' > t$)
    \If{all primal constraints corresponding to time $t$ are satisfied}
        \State Increase variable $y(t,S)$ continuously
        \State For each $p \in S \backslash \{p_t\}$, increase the variable $x(p,r(p,t))$ at rate
        \begin{align*}
            \frac{ \d x(p, r(p,t)) }{ \d y (t,S) } = \frac{1}{c_p} \cdot \ln (k+1) \cdot \min \{ w(S) - k, w_p \} \cdot ( x(p, r(p,t)) + 1/k )
        \end{align*}
        \State If $x(p, r(p,t)) = 1$, then remove $p$ from $S$, i.e., $S \leftarrow S \backslash \{p \}$
    \Else
        \State
    \EndIf
\EndProcedure
\end{algorithmic}
\end{algorithm}

\end{document}
